\documentclass{article}
\usepackage[utf8]{inputenc}
\usepackage{geometry}
\usepackage{amssymb}
\usepackage{amsmath}
\usepackage{listings}
\usepackage{array}
\usepackage{calc}

\usepackage{tikz}

\usepackage{biblatex} %Imports biblatex package
\addbibresource{ref.bib} %Import the bibliography file

\usepackage[linesnumbered]{algorithm2e}

\geometry{hmargin=4cm,vmargin=1.5cm}

\title{Internship Report}
\author{Solal Rapaport }
\date{June 2022}

\begin{document}

\maketitle

\section{Introduction}

There are two goals here, the first one is to build formulas that will allow robots, spread on a ring, to gather. We have $k$ robots and we will use view vectors to build those formulas. The formulas will be an interpretation of the pseudo-code given in the research report \cite{gathering}.

The formulas we are building, will be used with formulas given in an other research report \cite{algo}, and then will be tested in the acceleration algorithm using an interpolant \cite{algo}. Which leads us to the second goal, we want to implement and, if possible, improve this algorithm.

\section{Logical Formulas}

In this section, we will translate the algorithms given in the research report \cite{gathering}. Some changes will have to be made because we can't literally translate an algorithm into a first-order logic formula.

Before each formula we will describe briefly their scope : when will they be true (or false). We won't present to you the implementation of those formulas in this report. There will be an annex available with the \textit{Python} implementation that we use in order to test those formulas and to put them in the algorithm \cite{algo}.

We have three strategies. Each of them allows a robot to move in a given direction based on its environment. They all have the same definition, they take one argument : the view vector (distance vector).

\subsection{Configurations with single multiplicity}

The strategy $\phi_{SM}$ is $true$ if the given configuration has a single multiplicity and that the robot calling the strategy should move toward the robot at distance $d_{0}$ :
\begin{center}
$\phi_{SM}(d_0, \ldots , d_{k-1}):=$\\
$(\bigvee_{i=0}^{k-1}(d_i = 0\bigwedge_{j=0\ j\not=i}^{k-1} (d_j > 0 \lor (d_j = 0 \land d_{j-1} = 0) ) ))\land$\\
$(d_{k-1} \not= 0) \land $\\
$((d_1 = 0 \land d_{k-2} = 0 \land d_0 \leq d_{k-1}) \lor (d_1 = 0 \land d_{k-2} \not= 0))$
\end{center}

In order to test our strategy we need a function that will initialize our first configuration and make it one with a single multiplicity without being already a winning one, we just thought it'd be a neat thing to do. Here is the formula $InitSM$ which is $true$ if $p$, $s$ and $t$ form a configuration with a single multiplicity, the configuration is not a winning one, all $p$ are initialized in the right scope, all $t$ are initialized at $0$ and all $s$ are initialized at $RLC$ ($-1$) :

\begin{center}
    
$InitSM(p_{0},\dots, p_{k-1}, s_{0}, \dots, s_{k-1}, t_{0}, \dots, t_{k-1}, size_{ring}):=$\\
$\bigvee_{i = 0}^{k-1}( p_{i} \not= p_{i+1 \mod{k-1}} ) \land$\\
$(\bigwedge_{i = 0}^{k-1} ( p_{i} \geq 0 \land p_{i} <  size_{ring} \land s_{i} = -1 \land t_{i} = 0)) \land$\\
$(\bigvee_{i = 0}^{k-1} (\bigvee_{j = 0, j\not=i}^{k-1} (p_{j} = p_{i} \land \bigwedge_{h = 0}^{k-1} ( \bigwedge_{l = 0, l\not=h}^{k-1} (p_{h} \not= p_{l} \lor p_{h} = p_{i}) ) ) ) )$
\end{center}

\begin{figure}
    \centering
    \def\svgscale{0.3}
    \input{dessinSM.pdf_tex}
    \caption{Single multiplicity configuration. Here, the view vector or C is (4, 0, 5, 3). Because there's only one 0 we know there is a single multiplicity. Because the 0 isn't the last int of the vector we know C is not on the multiplicity. There's only one free segment toward the multiplicity, hence C can move on this segment.}
    \label{dessinSM}
\end{figure}

\subsection{Gathering rigid configurations}

Let $d_{ij}$ be the value $j$ of the view vector of the robot $i$, and $ds_{ij}$ the value $j$ of the symmetrical 
view of the robot $i$.
The robot is calling the strategy $\phi_R$.

Here are all the logic formulas used in order to build $\phi_R$:\newline


$AllView$ is $true$ if $d_{00}, \ldots ,d_{k-1k-1}$ are all the views you can obtain from a single view vector $dist_{0}, \ldots ,dist_{k-1}$ :

\begin{center}
    
$AllView(dist_{0}, \ldots ,dist_{k-1}, d_{00}, \ldots ,d_{k-1k-1}):=$\\
$(\bigwedge_{i=0}^{k-1} (\bigwedge_{j=0}^{k-1} (d_{ij} = dist_{(j+i) \mod{k}}) ) )$
\end{center}

$IsRigid$ is $true$ if the given configuration is a rigid configuration. Meaning, all views are distinct, there is no multiplicity, and the configuration isn't symmetric nor periodic.

\begin{center}

$IsRigid(d_{00}, \ldots ,d_{k-1k-1}, ds_{00}, \ldots ,ds_{k-1k-1}):=$\\
$\bigwedge_{i=0}^{k-1}(\bigwedge_{j=0}^{k-1}d_{ij}\not=0)\land $\\%pas de multiplicité
$\bigwedge_{i=0}^{k-1}(
\bigwedge_{l=0\ l\not=i}^{k-1}(
(\bigvee_{j=0}^{k-1}d_{ij} \not= d_{lj})
\land (\bigvee_{j=0}^{k-1}d_{ij} \not= ds_{lj})$\\
$\land (\bigvee_{j=0}^{k-1}ds_{ij} \not= d_{lj})
\land (\bigvee_{j=0}^{k-1}ds_{ij} \not= ds_{lj})
) )$\\%Toutes les vues sont distinctes
\end{center}

$AllCode$ is $true$ if $(\alpha'_{r}, \beta'_{r})$ is the set of two natural numbers of the robot $r$ such as $\alpha'_r$ and $\beta'_r$ are codes of $r$'s views, with $\alpha'_{r} < \beta'_{r}$. The process which leads us to obtain all view codes is defined in the research report \cite{gathering}. 

\begin{center}
$AllCode(d_{00}, \ldots ,d_{k-1k-1}, ds_{00}, \ldots ,ds_{k-1k-1}, \alpha_{0}, \dots, \alpha_{k-1}, \beta_{0}, \dots, \beta_{k-1},$\\
$\alpha'_{0}, \dots, \alpha'_{k-1}, \beta'_{0}, \dots, \beta'_{k-1}) :=$\\
$\bigwedge_{i = 0}^{k-1} \left( \alpha'_{i} < \beta'_{i} \land (\alpha'_{i} = \alpha_{i} \lor \alpha'_{i} = \beta_{i}) \land (\beta'_{i} = \alpha_{i} \lor \beta'_{i} = \beta_{i}) \right) \land $\\
$($
$(\alpha_{0} < \alpha_{1} < \dots < \alpha_{k-1} < \beta_{0} < \dots < \beta_{k-1}) \land $\\
$(\bigvee_{p=0}^{k-1}(\bigwedge_{q=0}^{p-1}(d_{0q} = d_{1q}) \land d_{0p} > d_{1p} ) )\land \dots \land $\\
$(\bigvee_{p=0}^{k-1}(\bigwedge_{q=0}^{p-1}(ds_{(k-2)q} = ds_{(k-1)q}) \land ds_{(k-2)p} > ds_{(k-1)p} ) )$\\
$\lor $\\
$((\alpha_{0} < \alpha_{2} < \alpha_{1} < \dots < \alpha_{k-1} < \beta_{0} < \dots < \beta_{k-1}) \land \dots) \lor \dots$
$)$
\end{center}

$CodeMaker$ is $true$ if the configuration is rigid and if $(a_{0}, \ldots , a_{k-1}, as_{0}, \ldots , as_{k-1})$ are each code of each view passed as a parameter :

\begin{center}
$CodeMaker(d_{00}, \ldots ,d_{k-1k-1}, ds_{00}, \ldots ,ds_{k-1k-1}, a_{0}, \ldots , a_{k-1}):=$\\
$IsRigid(d_{00}, \ldots ,d_{k-1k-1}, ds_{00}, \ldots ,ds_{k-1k-1}) \land $\\
%   le couple (alpha', beta') le premier ordre lexicographique, couple d'entier ordonné
%   alpha code vue, beta code de vue symétrique
$\exists \alpha_{0}, \dots, \alpha_{k-1}, \beta_{0}, \dots, \beta_{k-1},$
$\alpha'_{0}, \dots, \alpha'_{k-1}, \beta'_{0}, \dots, \beta'_{k-1},$\\
$AllCode(d_{00}, \ldots ,d_{k-1k-1}, ds_{00}, \ldots ,ds_{k-1k-1}, \alpha_{0}, \dots, \alpha_{k-1}, \beta_{0}, \dots, \beta_{k-1}, $\\
$\alpha'_{0}, \dots, \alpha'_{k-1}, \beta'_{0}, \dots, \beta'_{k-1})$\\
$(\bigwedge_{i = 0}^{k-1} (\bigwedge_{j = 0, j\not=i}^{k-1} ((a_{i} > a_{j} \land \alpha'_{j} > \alpha'_{i}) \lor (a_{i} < a_{j} \land \alpha'_{j} < \alpha'_{i}) ) ) )$\\
$\bigwedge_{i=0}^{k-1} (\bigwedge_{j=0, j \not= i}^{k-1} a_{i} \not= a_{j})$
\end{center}

$FindMax$ is $true$ if $Max$ is the highest value of the view vector passed as a parameter :

\begin{center}

$FindMax(dist_{0}, \ldots ,dist_{k-1}, Max):=$\\
$(\bigwedge_{i=0}^{k-1} (Max \geq dist_{i}) \land (\bigvee_{i=0}^{k-1} (Max = dist_{i})))$
\end{center}

$FindM$ is $true$ if $M$ is the index of the robot (index in the view vector) which has the largest code of view and a neighboring robot at distance $Max$ :

\begin{center}

$FindM(d_{00}, \ldots ,d_{k-1k-1}, a_{0}, \ldots , a_{k-1}, Max, dM_{0}, \dots, dM_{k-1}):=$\\
$\bigvee_{m=0}^{k-1}((\bigwedge_{i=0}^{k-1} ((a_{m} \geq a_i \land (d_{i0} = Max \lor d_{ik-1} = Max))$\\
$\lor (d_{i0} < Max \land d_{ik-1} < Max))) \land M = m )$
\end{center}

$FindN$ is $true$ if $N$ is the index of the robot (index in the view vector) with the largest code of view and $M$ as a neighboring robot at distance $Max$ :

\begin{center}

$FindN(d_{00}, \ldots ,d_{k-1k-1}, a_{0}, \ldots , a_{k-1}, Max, M, N):=$\\
$(d_{M0} = Max \land d_{Mk-1} = Max \land $\\
$( (N = ((M+1) \mod{k}) \land a_{(M+1) \mod{k}} > a_{(M-1) \mod{k}}) \lor $\\
$(N = ((M-1) \mod{k}) \land a_{(M-1) \mod{k}} > a_{(M+1) \mod{k}})))$\\
$ \lor$\\
$(d_{M0} = Max \land d_{Mk-1} \not= Max \land N = ((M+1)\mod{k}))$\\
$ \lor$\\
$(d_{M0} \not= Max \land d_{Mk-1} = Max \land N = ((M-1)\mod{k}))$
\end{center}

Since those formulas can't be implemented in \textit{Python} because it is impossible to work around a variable index, we choose to build a new formula, $FindMN$ that will be $true$ if both vectors $dM$ and $dN$ are the view vector of, respectively, $M$ and $N$.

\begin{center}

$FindMN(d_{00}, \ldots ,d_{k-1k-1}, a_{0}, \ldots , a_{k-1}, Max, M, N,$\\
$ dM_{0}, \dots, dM_{k-1}, dN_{0}, \dots, dN_{k-1}):=$\\
$\bigvee_{m=0}^{k-1}(\quad
(\bigwedge_{i=0}^{k-1} ((a_{m} \geq a_i \land (d_{i0} = Max \lor d_{ik-1} = Max))$\\
$\lor (d_{i0} < Max \land d_{ik-1} < Max))) \land M = m \land $\\ 
%(\bigwedge_{l = 0}^{k-1} (dM_{l} = d_{ml}) )
%(\bigwedge_{l = 0}^{k - 1} (dN_{l} = d_{(m+1 \mod{k}) l}) )
%(\bigwedge_{l = 0}^{k - 1} (dN_{l} = d_{(m-1 \mod{k}) l}) )
$(\quad (
d_{m0} = Max \land d_{mk-1} = Max \land $\\
$( ( N = M+1 \mod k \land a_{(m+1) \mod{k}} > a_{(m-1) \mod{k}}) \lor $\\
$( N = M-1 \mod{k} \land a_{(m-1) \mod{k}} > a_{(m+1) \mod{k}}))
)\lor $\\
$(
d_{m0} = Max \land d_{mk-1} \not= Max \land N = M+1 \mod k
) \lor$\\
$(
d_{m0} \not= Max \land d_{mk-1} = Max \land N = M-1 \mod{k}
)\quad ) \land $\\
$( (N = M-1\mod{k} \land (\bigwedge_{l = 0}^{k-1} (dN_{l} = d_{(m-1 \mod{k}) ((k-1)-l)} \land dM_{l} = d_{ml}) ) ) \lor $\\
$(N = M+1\mod{k} \land (\bigwedge_{l = 0}^{k-1} (dN_{l} = d_{(m+1 \mod{k})l} \land dM_{l} = d_{m((k-1)-l)}) ) ) )$
$\quad )$
\end{center}

$\phi_R$ is $true$ if the configuration is rigid, and if the robot is $M$ and has a closest neighbor than $N$, or if the robot is $N$ and has a closest neighbor than $M$.

\begin{center}

$\phi_R(dist_{0}, \ldots ,dist_{k-1}):=$\\
$\exists d_{00}, \ldots ,d_{k-1k-1},\ AllView(dist_{0}, \ldots ,dist_{k-1}, d_{00}, \ldots ,d_{k-1k-1})\land$\\
$\exists ds_{00}, \ldots ,ds_{k-1k-1}, \bigwedge_{i=0}^{k-1} (ViewSym(d_{i0}, \ldots , d_{ik-1}, ds_{i0}, \ldots , ds_{ik-1}))\land$\\
$\exists Max, a_{0}, \ldots , a_{k-1}, dM_{0}, \dots, dM_{k-1}, dN_{0}, \dots, dN_{k-1},$\\
%on fait les codes de vues
$CodeMaker(d_{00}, \ldots ,d_{k-1k-1}, ds_{00}, \ldots ,ds_{k-1k-1}, a_{0}, \ldots , a_{k-1})\land$\\
%On initialise Max, M et N
$FindMax(dist_{0}, \ldots ,dist_{k-1}, Max) \land$\\
%Initialiser M et N
$FindMN(d_{00}, \ldots ,d_{k-1k-1}, a_{0}, \ldots , a_{k-1}, Max,  dM_{0}, \dots, dM_{k-1}, dN_{0}, \dots, dN_{k-1}) \land $\\
$\exists dM2_{0}, \dots, dM2_{k-1}, dN2_{0}, \dots, dN2_{k-1},$\\
$( (\bigwedge_{i = 0}^{k-1} (dM2_{i} = dM_{i+1 \mod{k}}) ) \lor (\bigwedge_{i = 0}^{k-1} (dM2_{i} = dM_{i-1 \mod{k}})) ) \land$\\
$(\bigvee_{i = 0}^{k-1} (dM2_{i} \not= dN_{i}) ) \land $\\
$( (\bigwedge_{i = 0}^{k-1} (dN2_{i} = dN_{i+1 \mod{k}}) ) \lor (\bigwedge_{i = 0}^{k-1} (dN2_{i} = dN_{i-1 \mod{k}})) ) \land$\\
$(\bigvee_{i = 0}^{k-1} (dN2_{i} \not= dM_{i}) ) \land $\\

%On initialise toutes les distances possibles par rapport à M et N
$\exists distM_{0}, \ldots , distM_{k-1}, distN_{0}, \ldots , distN_{k-1},$\\
$\bigwedge_{i=0}^{k-1}(distM_{i} = (\sum_{l=0}^i dM_{l}) \land distN_{i} = (\sum_{l=0}^i dN_{l}) ) \land $\\
$(\bigvee_{i=0}^{k-1}\quad (\quad (distM_{i} < distN_{i} \bigwedge_{q=0}^{i} (distM_{q} = distN_{q}) \bigwedge_{j = 0}^{k-1} (dM_{j} = dist_{j}) ) \lor $\\
$(distM_{i} > distN_{i} \bigwedge_{q=0}^{i} (distM_{q} = distN_{q}) \bigwedge_{j = 0}^{k-1} (dN_{j} = dist_{j}) )\quad )\quad )$
\end{center}

\subsection{Gathering an odd number of robots}

We are now building a strategy, $\phi_{ON}$, that will gather an odd number of robots on a non-periodic configuration. It is the strategy with the lowest priority, meaning that the configuration won't be rigid and won't have any multiplicity.

First we build the formula, $IsPeriodic$, that will return $true$ if the configuration is periodic with an odd number of robots :

\begin{center}
    
$IsPeriodic(dist_{0}, \ldots , dist_{k-1}):=$\\
$\exists p \in [1; \lfloor \frac{k}{3} \rfloor ], (p+1) \mod{2} = 0 \land $\\
$\exists d'_{0}, \ldots , d'_{p-1}, \bigwedge_{i=0}^{k-1} (d'_{i\mod{p}} = dist_{i})$
\end{center}

Now, we build build $\phi_{OD}$, the strategy returns $true$ if the configuration is non-rigid, non-periodic, has no multiplicity and has an odd number of robots. If the robot is axial then it moves in order to create a multiplicity or a rigid configuration.

\begin{center}
    
$\phi_{ON}(dist_{0}, \ldots , dist_{k-1}):=$\\
$\exists d_{00}, \ldots ,d_{k-1k-1},\ AllView(dist_{0}, \ldots ,dist_{k-1}, d_{00}, \ldots ,d_{k-1k-1})\land$\\
$\exists ds_{00}, \ldots ,ds_{k-1k-1}, \bigwedge_{i=0}^{k-1} (ViewSym(d_{i0}, \ldots , d_{ik-1}, ds_{i0}, \ldots , ds_{ik-1}))\land$\\
$\lnot IsRigid(d_{00}, \ldots ,d_{k-1k-1}, ds_{00}, \ldots ,ds_{k-1k-1}) \land $\\
$((k+1) \mod{2} = 0) \land $\\
$\lnot IsPeriodic(dist_{0}, \ldots , dist_{k-1}) \land $\\
$(\bigwedge_{i=0}^{k-1} dist_{i} \not= 0) \land $\\
$(\bigwedge_{i=0}^{k-1} dist_{i} = ds_{0i})$
\end{center}

\section{Algorithms}

Now that we have done all of our logical formulas, we need to test those in the acceleration algorithm using an interpolant \cite{algo} and in an alternate version of that same algorithm.

We needed to create an alternate version because of the way the formula, $BouclePerdante$, is done. Two ways it can be done :
\begin{enumerate}
    \item we can try to create a loosing loop by trying to add as many $AsyncPost$ as needed (increase the size of the loop if it's not a loosing one) with a maximum of the size of the graph of all possible configurations
    \item or we can try to create a loop that comes back to a previous configuration with only one $AsyncPost$
\end{enumerate}

The first possibility has been implemented in the acceleration algorithm using an interpolant \cite{algo}. In order to implement the second possibility we needed to change the algorithm because the winning condition wasn't good anymore.

First we will try to prove that the alternate version of the algorithm works.
Here is the algorithm :

\SetKwComment{Comment}{/* }{ */}
\begin{algorithm}
\ForEach{synchronous winning strategy $f$}{
    $k = 1$\;
    \While{$true$}{
        $I(c) = Init(c)$\;
        $continue = true$\;
        \While{continue}{
            \If{$MaybeThisSize \not= null$}{
                $NotThisSizeBis = $[$i$ for $i$ in range($k$) and $i \notin MaybeThisSize$]\;
                \If{$Init(c) \land Post(c, c1), Post(c1, c2) \land \dots \land Post(c_{k-1}, c_{k}) \land BouclePerdante(c_{k}, NotThisSizeBis)\quad SAT$}{
                    $exit$\Comment*[r]{Loosing Strategy}
                }
            }
            \eIf{$I(c) \land Post(c, c1), Post(c1, c2) \land \dots \land Post(c_{k-1}, c_k) \land BouclePerdante(c_k, NotThisSize)\quad SAT$}{
                \eIf{$I = Init$}{
                    $exit$\Comment*[r]{Loosing Strategy}
                }{
                    $MaybeThisSize.append(k)$\;
                    $k = k + 1$\;
                    $continue = false$\;
                }
            }{
                $I' = Interpolant(I(c) \land Post(c, c1), Post(c1, c2) \land \dots \land Post(c_{k-1}, c_k) \land BouclePerdante(c_k, NotThisSize))$\;
                \eIf{$I' \implies I$}{
                    \eIf{$k = size_{max}$}{
                        $exit$\Comment*[r]{Winning Strategy}
                    }{
                        $NotThisSize.append(k)$\;
                        $k = k + 1$\;
                        $continue = false$\;
                    }
                }{
                    $I = I \lor I'$\;
                }
            }
        }
    }
}
\end{algorithm}
\newpage

\noindent \textbf{Proof :}\newline

First let's talk about the termination of the algorithm :
\begin{itemize}
    \item the list of synchronous winning strategy is finished
    \item we can exit the "\textbf{while} $true$" (l.3) loop with $exit$ instructions that we find at line 10, 15 and 25.
    \begin{itemize}
        \item we find a loosing loop without the interpolant and then we enter the $exit$ at line 10 or the one at line 15 if $I$ is still equal to $Init$
        \item we find a loosing loop with the interpolant and then we increase $k$, we exit the "\textbf{while} $continue$" loop (l.6) which allows us to reinitialize $I$ and test if a loosing loop exists for a higher $k$ or for this $k$ without the interpolant.
        \item we don't find any loosing loop, then, eventually, the interpolant will stop growing and $(I \lor I') \implies I$, likewise, $k$ will reach $size_{max}$ and we will enter the $exit$ at line 25. $k$ will always reach $size_{max}$ if there is no loosing loop, because if the condition line 13, which checks if there is a loosing loop, is false, then if $k < size_{max}$ we reached line 28 and we increase $k$. Also, the interpolant will eventually stop growing because the graph of all possible configurations is finished and the interpolant won't create new variables.
    \end{itemize}
    \item to summarize, we can't have more than $size_{max}$ failure at finding a loosing loop and if we find one we either exit if $Init = I$ or we keep trying until we find none or one where $Init = I$.
    \item TODO
\end{itemize}

\section{Tests}

In order to test the algorithm \cite{algo} we will use the python code we show you at the beginning : $InitSM$ and $phiSM$.

We will use the SAT-solver to test different configurations. We will change the number of robots and the size of the ring from a test to an other.

\subsection{Test $InitSM$}

First, we test the function $InitSM$ alone : can we have an initial configuration with a single multiplicity with those parameters ?\\

\begin{tabular}{|r|c|c|c|c|c|}
  \hline
  nb-robot \textbackslash size-ring & 2 & 3 & 4 & 5 & 6\\
  \hline
  2 & Unsat & Unsat & Unsat & Unsat & Unsat \\
  \hline
  3 & Sat & Sat & Sat & Sat & Sat \\
  \hline
  4 & Sat & Sat & Sat & Sat & Sat \\
  \hline
  5 & Sat & Sat & Sat & Sat & Sat \\
  \hline
  6 & Sat & Sat & Sat & Sat & Sat \\
  \hline
\end{tabular}
\\

The results make sense : we can't create a multiplicity with 2 robots which is not a winning configuration. Else, even on a ring size of 2 we can have a multiplicity on one spot and only one robot on the other spot.

\subsection{Test $\phi_{SM}$}

Now we test $\phi_{SM}$ through the algorithm \cite{algo}, we also use the function $InitSM$ that makes sure we have a single multiplicity at the beginning.\\

\begin{tabular}{|r|c|c|c|c|c|}
  \hline
  nb-robot \textbackslash size-ring & 2 & 3 & 4 & 5 & 6\\
  \hline
  3 & Timeout & Timeout & Timeout & ... & ... \\
  \hline
  4 & Timeout & ... & ... & ... & ... \\
  \hline
  5 & ... & ... & ... & ... & ... \\
  \hline
  6 & ... & ... & ... & ... & ... \\
  \hline
\end{tabular}
\\

Same test but with the function $Init$ instead.\\

\begin{tabular}{|r|c|c|c|c|c|}
  \hline
  nb-robot \textbackslash size-ring & 2 & 3 & 4 & 5 & 6\\
  \hline
  3 & Timeout & Loose & Loose & Loose & ... \\
  \hline
  4 &  & ... & ... & ... & ... \\
  \hline
  5 & ... & ... & ... & ... & ... \\
  \hline
  6 & ... & ... & ... & ... & ... \\
  \hline
\end{tabular}

For $nb_{robot} = 3$ and $size_{ring} = 2$ we face this problem :
\begin{lstlisting}
Traceback (most recent call last):

  File "algov5.py", line 56, in <module>
  
    Ip = tree_interpolant(And(Interpolant(And(tmpAndInterpolant)), 
    And(tmpAndContext)))
    
  File "/usr/lib/python3.8/site-packages/z3/z3.py", line 8297, 
  in tree_interpolant
  
    res = Z3_compute_interpolant(ctx.ref(),f.as_ast(),p.params,ptr,mptr)
    
  File "/usr/lib/python3.8/site-packages/z3/z3core.py", line 4074, 
  in Z3_compute_interpolant
  
    _elems.Check(a0)
    
  File "/usr/lib/python3.8/site-packages/z3/z3core.py", line 1336, in Check
  
    raise self.Exception(self.get_error_message(ctx, err))
    
z3.z3types.Z3Exception: b'theory not supported by interpolation or bad proof'
\end{lstlisting}

\newpage

\printbibliography %Prints bibliography
\end{document}